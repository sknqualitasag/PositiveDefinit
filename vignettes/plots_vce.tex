\documentclass[]{article}
\usepackage{lmodern}
\usepackage{amssymb,amsmath}
\usepackage{ifxetex,ifluatex}
\usepackage{fixltx2e} % provides \textsubscript
\ifnum 0\ifxetex 1\fi\ifluatex 1\fi=0 % if pdftex
  \usepackage[T1]{fontenc}
  \usepackage[utf8]{inputenc}
\else % if luatex or xelatex
  \ifxetex
    \usepackage{mathspec}
  \else
    \usepackage{fontspec}
  \fi
  \defaultfontfeatures{Ligatures=TeX,Scale=MatchLowercase}
\fi
% use upquote if available, for straight quotes in verbatim environments
\IfFileExists{upquote.sty}{\usepackage{upquote}}{}
% use microtype if available
\IfFileExists{microtype.sty}{%
\usepackage{microtype}
\UseMicrotypeSet[protrusion]{basicmath} % disable protrusion for tt fonts
}{}
\usepackage[margin=1in]{geometry}
\usepackage{hyperref}
\hypersetup{unicode=true,
            pdftitle={Plot results of VCE},
            pdfborder={0 0 0},
            breaklinks=true}
\urlstyle{same}  % don't use monospace font for urls
\usepackage{color}
\usepackage{fancyvrb}
\newcommand{\VerbBar}{|}
\newcommand{\VERB}{\Verb[commandchars=\\\{\}]}
\DefineVerbatimEnvironment{Highlighting}{Verbatim}{commandchars=\\\{\}}
% Add ',fontsize=\small' for more characters per line
\usepackage{framed}
\definecolor{shadecolor}{RGB}{248,248,248}
\newenvironment{Shaded}{\begin{snugshade}}{\end{snugshade}}
\newcommand{\KeywordTok}[1]{\textcolor[rgb]{0.13,0.29,0.53}{\textbf{#1}}}
\newcommand{\DataTypeTok}[1]{\textcolor[rgb]{0.13,0.29,0.53}{#1}}
\newcommand{\DecValTok}[1]{\textcolor[rgb]{0.00,0.00,0.81}{#1}}
\newcommand{\BaseNTok}[1]{\textcolor[rgb]{0.00,0.00,0.81}{#1}}
\newcommand{\FloatTok}[1]{\textcolor[rgb]{0.00,0.00,0.81}{#1}}
\newcommand{\ConstantTok}[1]{\textcolor[rgb]{0.00,0.00,0.00}{#1}}
\newcommand{\CharTok}[1]{\textcolor[rgb]{0.31,0.60,0.02}{#1}}
\newcommand{\SpecialCharTok}[1]{\textcolor[rgb]{0.00,0.00,0.00}{#1}}
\newcommand{\StringTok}[1]{\textcolor[rgb]{0.31,0.60,0.02}{#1}}
\newcommand{\VerbatimStringTok}[1]{\textcolor[rgb]{0.31,0.60,0.02}{#1}}
\newcommand{\SpecialStringTok}[1]{\textcolor[rgb]{0.31,0.60,0.02}{#1}}
\newcommand{\ImportTok}[1]{#1}
\newcommand{\CommentTok}[1]{\textcolor[rgb]{0.56,0.35,0.01}{\textit{#1}}}
\newcommand{\DocumentationTok}[1]{\textcolor[rgb]{0.56,0.35,0.01}{\textbf{\textit{#1}}}}
\newcommand{\AnnotationTok}[1]{\textcolor[rgb]{0.56,0.35,0.01}{\textbf{\textit{#1}}}}
\newcommand{\CommentVarTok}[1]{\textcolor[rgb]{0.56,0.35,0.01}{\textbf{\textit{#1}}}}
\newcommand{\OtherTok}[1]{\textcolor[rgb]{0.56,0.35,0.01}{#1}}
\newcommand{\FunctionTok}[1]{\textcolor[rgb]{0.00,0.00,0.00}{#1}}
\newcommand{\VariableTok}[1]{\textcolor[rgb]{0.00,0.00,0.00}{#1}}
\newcommand{\ControlFlowTok}[1]{\textcolor[rgb]{0.13,0.29,0.53}{\textbf{#1}}}
\newcommand{\OperatorTok}[1]{\textcolor[rgb]{0.81,0.36,0.00}{\textbf{#1}}}
\newcommand{\BuiltInTok}[1]{#1}
\newcommand{\ExtensionTok}[1]{#1}
\newcommand{\PreprocessorTok}[1]{\textcolor[rgb]{0.56,0.35,0.01}{\textit{#1}}}
\newcommand{\AttributeTok}[1]{\textcolor[rgb]{0.77,0.63,0.00}{#1}}
\newcommand{\RegionMarkerTok}[1]{#1}
\newcommand{\InformationTok}[1]{\textcolor[rgb]{0.56,0.35,0.01}{\textbf{\textit{#1}}}}
\newcommand{\WarningTok}[1]{\textcolor[rgb]{0.56,0.35,0.01}{\textbf{\textit{#1}}}}
\newcommand{\AlertTok}[1]{\textcolor[rgb]{0.94,0.16,0.16}{#1}}
\newcommand{\ErrorTok}[1]{\textcolor[rgb]{0.64,0.00,0.00}{\textbf{#1}}}
\newcommand{\NormalTok}[1]{#1}
\usepackage{graphicx,grffile}
\makeatletter
\def\maxwidth{\ifdim\Gin@nat@width>\linewidth\linewidth\else\Gin@nat@width\fi}
\def\maxheight{\ifdim\Gin@nat@height>\textheight\textheight\else\Gin@nat@height\fi}
\makeatother
% Scale images if necessary, so that they will not overflow the page
% margins by default, and it is still possible to overwrite the defaults
% using explicit options in \includegraphics[width, height, ...]{}
\setkeys{Gin}{width=\maxwidth,height=\maxheight,keepaspectratio}
\IfFileExists{parskip.sty}{%
\usepackage{parskip}
}{% else
\setlength{\parindent}{0pt}
\setlength{\parskip}{6pt plus 2pt minus 1pt}
}
\setlength{\emergencystretch}{3em}  % prevent overfull lines
\providecommand{\tightlist}{%
  \setlength{\itemsep}{0pt}\setlength{\parskip}{0pt}}
\setcounter{secnumdepth}{0}
% Redefines (sub)paragraphs to behave more like sections
\ifx\paragraph\undefined\else
\let\oldparagraph\paragraph
\renewcommand{\paragraph}[1]{\oldparagraph{#1}\mbox{}}
\fi
\ifx\subparagraph\undefined\else
\let\oldsubparagraph\subparagraph
\renewcommand{\subparagraph}[1]{\oldsubparagraph{#1}\mbox{}}
\fi

%%% Use protect on footnotes to avoid problems with footnotes in titles
\let\rmarkdownfootnote\footnote%
\def\footnote{\protect\rmarkdownfootnote}

%%% Change title format to be more compact
\usepackage{titling}

% Create subtitle command for use in maketitle
\providecommand{\subtitle}[1]{
  \posttitle{
    \begin{center}\large#1\end{center}
    }
}

\setlength{\droptitle}{-2em}

  \title{Plot results of VCE}
    \pretitle{\vspace{\droptitle}\centering\huge}
  \posttitle{\par}
    \author{}
    \preauthor{}\postauthor{}
      \predate{\centering\large\emph}
  \postdate{\par}
    \date{2019-10-02}


\begin{document}
\maketitle

\section{Overview of VCE results}\label{overview-of-vce-results}

\includegraphics{plots_vce_files/figure-latex/unnamed-chunk-3-1.pdf}

\includegraphics{plots_vce_files/figure-latex/unnamed-chunk-4-1.pdf}

\includegraphics{plots_vce_files/figure-latex/unnamed-chunk-5-1.pdf}

\section{Session Info}\label{session-info}

\begin{Shaded}
\begin{Highlighting}[]
\KeywordTok{sessionInfo}\NormalTok{()}
\CommentTok{#> R version 3.6.0 (2019-04-26)}
\CommentTok{#> Platform: x86_64-pc-linux-gnu (64-bit)}
\CommentTok{#> Running under: Ubuntu 18.04 LTS}
\CommentTok{#> }
\CommentTok{#> Matrix products: default}
\CommentTok{#> BLAS:   /usr/lib/x86_64-linux-gnu/blas/libblas.so.3.7.1}
\CommentTok{#> LAPACK: /usr/lib/x86_64-linux-gnu/lapack/liblapack.so.3.7.1}
\CommentTok{#> }
\CommentTok{#> locale:}
\CommentTok{#>  [1] LC_CTYPE=de_CH.UTF-8       LC_NUMERIC=C              }
\CommentTok{#>  [3] LC_TIME=de_CH.UTF-8        LC_COLLATE=de_CH.UTF-8    }
\CommentTok{#>  [5] LC_MONETARY=de_CH.UTF-8    LC_MESSAGES=de_CH.UTF-8   }
\CommentTok{#>  [7] LC_PAPER=de_CH.UTF-8       LC_NAME=C                 }
\CommentTok{#>  [9] LC_ADDRESS=C               LC_TELEPHONE=C            }
\CommentTok{#> [11] LC_MEASUREMENT=de_CH.UTF-8 LC_IDENTIFICATION=C       }
\CommentTok{#> }
\CommentTok{#> attached base packages:}
\CommentTok{#> [1] stats     graphics  grDevices utils     datasets  methods   base     }
\CommentTok{#> }
\CommentTok{#> loaded via a namespace (and not attached):}
\CommentTok{#>  [1] Rcpp_1.0.1            pillar_1.3.1          compiler_3.6.0       }
\CommentTok{#>  [4] plyr_1.8.4            bindr_0.1.1           tools_3.6.0          }
\CommentTok{#>  [7] digest_0.6.18         PositiveDefinit_0.1.0 evaluate_0.13        }
\CommentTok{#> [10] tibble_2.1.1          gtable_0.3.0          pkgconfig_2.0.2      }
\CommentTok{#> [13] rlang_0.4.0           yaml_2.2.0            xfun_0.6             }
\CommentTok{#> [16] bindrcpp_0.2.2        withr_2.1.2           stringr_1.4.0        }
\CommentTok{#> [19] dplyr_0.7.8           knitr_1.22            grid_3.6.0           }
\CommentTok{#> [22] tidyselect_0.2.5      glue_1.3.1            R6_2.3.0             }
\CommentTok{#> [25] rmarkdown_1.12        ggplot2_3.1.1         purrr_0.3.0          }
\CommentTok{#> [28] tidyr_0.8.3           reshape2_1.4.3        magrittr_1.5         }
\CommentTok{#> [31] scales_1.0.0          htmltools_0.3.6       assertthat_0.2.0     }
\CommentTok{#> [34] colorspace_1.4-1      labeling_0.3          stringi_1.4.3        }
\CommentTok{#> [37] lazyeval_0.2.2        munsell_0.5.0         crayon_1.3.4}
\end{Highlighting}
\end{Shaded}

\section{Latest Update}\label{latest-update}

2019-10-02 15:06:00 (zws)


\end{document}
